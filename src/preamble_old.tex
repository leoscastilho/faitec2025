%Referência de Modelo para a FAI: https://www.fai-mg.br/biblio/images/publicacoes/modelo/ModeloArtigoCientifico.pdf %

\usepackage[brazil]{babel}  % Usa o idioma português (Brasil) para definir as convenções de idioma, como a separação de sílabas e tradução de termos.
\usepackage[utf8]{inputenc}  % Define a codificação de entrada como UTF-8 para permitir caracteres acentuados corretamente.
\usepackage[T1]{fontenc}  % Define a codificação de saída de fontes como T1, permitindo o uso correto de caracteres especiais.
\usepackage{times}     % Fonte Times New Roman
\usepackage{parskip}   % Remove a identação dos parágrafos
\usepackage{geometry}  % Permite personalizar as margens e o layout da página.
\usepackage{graphicx}  % Permite inserir gráficos, imagens e ajustar o tamanho deles.
\usepackage{titlesec}  % Fornece comandos para personalizar o formato dos títulos das seções.
\usepackage{fancyhdr}  % Permite personalizar os cabeçalhos e rodapés do documento.
\usepackage{setspace}  % Controla o espaçamento entre linhas no documento (aqui é usado para espaçamento de 1,5 linha).
\usepackage{csquotes}  % Fornece suporte para citações, garantindo que o estilo de aspas seja correto em diferentes idiomas.
\usepackage{ragged2e}  % Fornece comandos para justificar ou ajustar o alinhamento de texto.
\usepackage{lipsum}    % Para gerar texto Lorem Ipsum
\usepackage{enumitem} % Customizar alíneas
\usepackage[toc,page]{appendix} % Para gerenciar apêndices
\usepackage{pgfmath} % Para cálculos matemáticos nos apêndices
\usepackage{tabularx}
\usepackage{calc}
\usepackage{textcase}

% Desabilitar hifenização
\tolerance=1
\emergencystretch=\maxdimen
\hyphenpenalty=10000
\hbadness=10000

\setlength{\parindent}{0pt}  % Paragráfos sem identação, conforme modelo da FAI
\setstretch{1.5}  % This also sets 1.5 line spacing
\setlength{\parskip}{12pt}  % Espaçament ode 12pt entre paragrafos

% Define a numeração da página
\fancypagestyle{numbered}{
    \fancyhf{}
    \fancyhead[R]{{\fontsize{10pt}{12pt}\selectfont\thepage}}
    \renewcommand{\headrulewidth}{0pt}
}

% Quanto à separação entre títulos, subtítulos e texto, observa-se o seguinte:
\let\oldsection\section
\renewcommand{\section}[1]{%
    \clearpage % os títulos de capítulos estão sempre no início de uma nova página;
    \thispagestyle{empty} % No page number on this page
    \oldsection{#1}
}
% seções primárias: os títulos dessas seções devem ser em caixa alta, fonte 12 e em negrito;
\titleformat{\section}{\normalfont\fontsize{12}{12}\selectfont\bfseries}{\thesection}{0.3em}{\MakeUppercase}

\let\oldsubsection\subsection
\renewcommand{\subsection}[1]{%
    \par\vspace{2\parskip}
    \oldsubsection{#1}
    \par\vspace{0.5\parskip} % separação entre o texto e os subtítulos: pula-se uma linha em branco.
}
% seções secundárias: os títulos dessas seções devem ser iguais aos das primárias, porém sem negrito;
\titleformat{\subsection}{\normalfont\fontsize{12}{12}\selectfont}{\thesubsection}{0.3em}{\MakeUppercase}

% Define spacing for subsubsection
\let\oldsubsubsection\subsubsection
\renewcommand{\subsubsection}[1]{%
    \par\vspace{2\parskip}
    \oldsubsubsection{#1}
    \par\vspace{0.5\parskip}% Add paragraph break after
}
% seções terciárias: os títulos dessas seções devem ser em negrito, somente com a primeira letra maiúscula;
\titleformat{\subsubsection}{\normalfont\fontsize{12}{14.4}\bfseries}{\thesubsubsection}{0.3em}{}

\geometry{
    left=3cm,
    right=2cm,
    top=3cm,
    bottom=2cm,
    headsep=35pt,
    footskip=0pt
    % , showframe
} % Define as margens da página (esquerda = 3cm, direita = 2cm, topo = 3cm, fundo = 2cm)

% Configuração das alíneas
\newlist{alinea}{enumerate}{1}
\setlist[alinea,1]{
    label=\alph*), % c) as alíneas devem ser indicadas alfabeticamente, em letra minúscula, seguida de parêntese fechado. Utilizam-se letras dobradas, quando esgotadas as letras do alfabeto;
    leftmargin=1cm,        % Total indentation from current margin
    itemindent=0pt,        % No additional item indentation
    labelsep=0pt,          % No space between label and text
    labelwidth=13pt,       % d) as letras indicativas das alíneas devem apresentar recuo de 0,5 cm em relação à margem esquerda;
    itemsep=0pt,
    parsep=0pt,
    topsep=0pt,
    partopsep=0pt,
    align=left,
    before={\setlength{\parskip}{0pt}},  % h) Entre as alíneas, inclusive a primeira delas, os parágrafos são formatados com zero ponto ante e zero depois
    after={\setlength{\parskip}{12pt}}   % i) O próximo parágrafo, após a última alínea, volta a ter formatação normal com 12 pontos antes e 12 depois.
}

\newcommand{\listaalinhada}[2]{%
  \noindent%
  \begin{tabularx}{\textwidth}{lX}
  {#1}: & \parbox[t]{\linewidth}{\raggedright #2}
  \end{tabularx}%
  \\
}

% Customização do Sumário ----------------------------------------------------------
\usepackage{tocloft}
\usepackage{xstring}
\usepackage{textcase}

\setcounter{tocdepth}{2} % Incluir apenas secao primaria e secundaria

\renewcommand{\contentsname}{SUMÁRIO}
\renewcommand{\cftsecfont}{\bfseries}
\renewcommand{\cftsecpagefont}{\bfseries}
\renewcommand{\cftsecleader}{\cftdotfill{\cftdotsep}}
\setlength{\cftbeforetoctitleskip}{0pt}
\setlength{\cftaftertoctitleskip}{0.4cm}
\renewcommand{\cfttoctitlefont}{\hfil\bfseries}


\renewcommand{\cfttoctitlefont}{\hfil\rmfamily\bfseries\MakeUppercase} % Título maiúsculo
\renewcommand{\cftdotsep}{0.1} % Adiciona mais pontos. Menor numero = Mais pontos
\setlength{\cftbeforesecskip}{0em} % Espaço entre as linhas de seção
\setlength{\cftbeforesubsecskip}{0em} % Espaço entre as linhas de subseção
\setlength{\cftbeforesubsubsecskip}{0em} % Espaço entre as linhas de subsubseção
\setlength{\cftsecindent}{0em} % Remove identação das seções
\setlength{\cftsubsecindent}{0em}   % Remove identação das subseções
\setlength{\cftsubsubsecindent}{0em} % Remove identação das subsubseções

\setlength{\cftsecnumwidth}{1.2em} % Espaço entre número da seção e texto
\setlength{\cftsubsecnumwidth}{1.5em} % Espaço entre número da subseção e texto
\setlength{\cftsubsubsecnumwidth}{2.3em} % Espaço entre número da subsubseção e texto

% Seção, subseção e subsubseção em maiúsculo no Sumário
\makeatletter
\let\oldcontentsline\contentsline
\renewcommand{\contentsline}[4]{%
  \expandafter\ifx\csname l@#1\endcsname\l@section
    \oldcontentsline{#1}{\MakeUppercase{#2}}{#3}{#4}%
  \else
    \expandafter\ifx\csname l@#1\endcsname\l@subsection
      \oldcontentsline{#1}{\MakeUppercase{#2}}{#3}{#4}%
    \else
      \expandafter\ifx\csname l@#1\endcsname\l@subsubsection
        \oldcontentsline{#1}{\MakeUppercase{#2}}{#3}{#4}%
      \else
        \oldcontentsline{#1}{#2}{#3}{#4}%
      \fi
    \fi
  \fi
}
\makeatother
% ------------------------------------------------------------------------------------

% Funções customizadas para criar e imprimir autores ---------------------------------
\usepackage{pgffor}

\newcounter{authorcount} % Counter para autores

% Comando para adicionar novos autores
\newcommand{\AdicionarAutor}[2]{% #1 = nome, #2 = email
    \stepcounter{authorcount}
    \expandafter\def\csname author\theauthorcount\endcsname{#1}
    \expandafter\def\csname email\theauthorcount\endcsname{#2}
}

% Comando para imprimir os autores
\newcommand{\imprimirautores}{%
    \foreach \i in {1,...,\theauthorcount} {%
        \textbf{\fontsize{12}{14}\selectfont\MakeUppercase{\csname author\i\endcsname}} \\
    }%
}

%  ------------------------------------------------------------------------
\newcounter{avaliadorcount}

\newcommand{\AdicionarBancaExaminadora}[2]{%
    \stepcounter{avaliadorcount}
    \expandafter\def\csname tipo\theavaliadorcount\endcsname{#1}
    \expandafter\def\csname nome\theavaliadorcount\endcsname{#2}
}

\newcommand{\imprimirbancaexaminadora}{%
    \foreach \i in {1,...,\theavaliadorcount} {%
        \underline{\hspace*{8cm}} \\
        \textbf{\fontsize{12}{14}\selectfont{\csname tipo\i\endcsname}} \textbf{\fontsize{12}{14}\selectfont{\csname nome\i\endcsname}} \\[2em]
    }%
}

% ------------------------------------------------------------------------------------

% Customização das Listas de Ilustrações ---------------------------------
\usepackage[labelfont=bf]{caption}
\usepackage{newfloat} % Para criar novos tipos de float

% Definir novos tipos de float
\DeclareFloatingEnvironment[
    fileext=fig,
    listname={Lista de Figuras},
    name=Figura,
    placement=htbp,
]{figura}

\DeclareFloatingEnvironment[
    fileext=fot,
    listname={Lista de Fotografias},
    name=Fotografia,
    placement=htbp,
]{fotografia}

\DeclareFloatingEnvironment[
    fileext=grf,
    listname={Lista de Gráficos},
    name=Gráfico,
    placement=htbp,
]{grafico}

\DeclareFloatingEnvironment[
    fileext=qua,
    listname={Lista de Quadros},
    name=Quadro,
    placement=htbp,
]{quadro}

% Contadores para cada tipo
\newcounter{totalfiguras}
\newcounter{totalfotografias}
\newcounter{totalgraficos}
\newcounter{totalquadros}

% Comandos para incrementar contadores
\newcommand{\addfigura}{\stepcounter{totalfiguras}}
\newcommand{\addfotografia}{\stepcounter{totalfotografias}}
\newcommand{\addgrafico}{\stepcounter{totalgraficos}}
\newcommand{\addquadro}{\stepcounter{totalquadros}}

% Sistema de duas passadas para listas condicionais
\makeatletter
% Escrever contadores no arquivo auxiliar ao final do documento
\AtEndDocument{%
    \immediate\write\@auxout{\string\gdef\string\@totalfiguras{\arabic{totalfiguras}}}%
    \immediate\write\@auxout{\string\gdef\string\@totalfotografias{\arabic{totalfotografias}}}%
    \immediate\write\@auxout{\string\gdef\string\@totalgraficos{\arabic{totalgraficos}}}%
    \immediate\write\@auxout{\string\gdef\string\@totalquadros{\arabic{totalquadros}}}%
}

% Definir valores padrão se não existirem (primeira compilação)
\@ifundefined{@totalfiguras}{\gdef\@totalfiguras{0}}{}
\@ifundefined{@totalfotografias}{\gdef\@totalfotografias{0}}{}
\@ifundefined{@totalgraficos}{\gdef\@totalgraficos{0}}{}
\@ifundefined{@totalquadros}{\gdef\@totalquadros{0}}{}

% Comandos condicionais usando os valores da compilação anterior
\newcommand{\ifhasfiguras}[2]{\ifnum\@totalfiguras>0 #1\else #2\fi}
\newcommand{\ifhasfotografias}[2]{\ifnum\@totalfotografias>0 #1\else #2\fi}
\newcommand{\ifhasgraficos}[2]{\ifnum\@totalgraficos>0 #1\else #2\fi}
\newcommand{\ifhasquadros}[2]{\ifnum\@totalquadros>0 #1\else #2\fi}
\makeatother
% ------------------------------------------------------------------------------------

% Customização da Lista de Figuras ---------------------------------
\usepackage[labelfont=bf]{caption}
\usepackage[figure]{totalcount} % Add this package
\makeatletter
\renewcommand{\@cftmakeloftitle}{} % Completely remove title generation
\makeatother

% Configure the figure entries with uppercase and dash
\renewcommand{\cftfigpresnum}{FIGURA }
\renewcommand{\cftfigaftersnum}{ -- }
\setlength{\cftfignumwidth}{2.5cm} % Adjust width for new format
\setlength{\cftfigindent}{0pt} % No indentation

% Define formatting for other illustration types
% The newfloat package creates \l@<environment> commands that we can customize
\makeatletter
% Custom formatting for quadro entries
\renewcommand*{\l@quadro}[2]{%
  \ifnum \c@tocdepth >\z@
    \addpenalty\@secpenalty
    \setlength\@tempdima{0.5cm}%
    \begingroup
      \parindent \z@ \rightskip \@pnumwidth
      \parfillskip -\@pnumwidth
      \leavevmode
      \advance\leftskip\@tempdima
      \hskip -\leftskip
      % --- The robust fix is here ---
      \def\numberline##1##2{%
        QUADRO ##1{} -- ##2%
      }%
      #1%
      \nobreak
      % --- End of fix ---
      \renewcommand{\@dotsep}{0.3}%
      \leaders\hbox{$\m@th\mkern \@dotsep mu\hbox{.}\mkern \@dotsep mu$}\hfill
      \nobreak\hb@xt@\@pnumwidth{\hss #2}\par
    \endgroup
  \fi}

% Custom formatting for fotografia entries  
% Custom formatting for fotografia entries
\renewcommand*{\l@fotografia}[2]{%
  \ifnum \c@tocdepth >\z@
    \addpenalty\@secpenalty
    \setlength\@tempdima{3cm}%
    \begingroup
      \parindent \z@ \rightskip \@pnumwidth
      \parfillskip -\@pnumwidth
      \leavevmode
      \advance\leftskip\@tempdima
      \hskip -\leftskip
      % --- The robust fix is here ---
      \def\numberline##1##2{%
        FOTOGRAFIA ##1 -- ##2%
      }%
      #1%
      \nobreak
      % --- End of fix ---
      \renewcommand{\@dotsep}{0.5}%
      \leaders\hbox{$\m@th\mkern \@dotsep mu\hbox{.}\mkern \@dotsep mu$}\hfill
      \nobreak\hb@xt@\@pnumwidth{\hss #2}\par
    \endgroup
  \fi}

% Custom formatting for grafico entries
\renewcommand*{\l@grafico}[2]{%
  \ifnum \c@tocdepth >\z@
    \addpenalty\@secpenalty
    \setlength\@tempdima{2.5cm}%
    \begingroup
      \parindent \z@ \rightskip \@pnumwidth
      \parfillskip -\@pnumwidth
      \leavevmode
      \advance\leftskip\@tempdima
      \hskip -\leftskip
      % --- The robust fix is here ---
      \def\numberline##1##2{%
        GRÁFICO ##1 -- ##2%
      }%
      #1%
      \nobreak
      % --- End of fix ---
      \renewcommand{\@dotsep}{0.5}%
      \leaders\hbox{$\m@th\mkern \@dotsep mu\hbox{.}\mkern \@dotsep mu$}\hfill
      \nobreak\hb@xt@\@pnumwidth{\hss #2}\par
    \endgroup
  \fi}
\makeatother

% Note: List formatting for other illustration types (fotografia, grafico, quadro)
% is handled by the newfloat package automatically. The caption formatting
% below will ensure consistent appearance in both captions and lists.
% ------------------------------------------------------------------------------------

\usepackage{caption}

\newcommand{\fonte}[1]{%
  \vspace{0em}%
  \begin{minipage}{\textwidth}%
    \raggedright%
    \footnotesize%
    FONTE: #1%
  \end{minipage}%
}

% Create \legenda as alias for \caption with full functionality
\NewCommandCopy{\legenda}{\caption}

% Create \equacao as alias for \equacao with full functionality
\newenvironment{equacao}{\begin{equation}}{\end{equation}}

% Custom caption formatting for all illustration types
% Using footnote size, capital letters, and dash separator
\captionsetup[quadro]{
    justification=raggedright,
    singlelinecheck=false,
    font={footnotesize},
    labelfont={footnotesize},
    labelformat=simple,
    labelsep=none,
    format=plain
}

\captionsetup[figura]{
    justification=raggedright,
    singlelinecheck=false,
    font={footnotesize},
    labelfont={footnotesize},
    labelformat=simple,
    labelsep=none,
    format=plain
}

\captionsetup[grafico]{
    justification=raggedright,
    singlelinecheck=false,
    font={footnotesize},
    labelfont={footnotesize},
    labelformat=simple,
    labelsep=none,
    format=plain
}

\captionsetup[fotografia]{
    justification=raggedright,
    singlelinecheck=false,
    font={footnotesize},
    labelfont={footnotesize},
    labelformat=simple,
    labelsep=none,
    format=plain
}

% Custom label formatting to use uppercase names with dashes
\DeclareCaptionLabelFormat{customdash}{\MakeUppercase{#1} #2 - }

% Apply custom label format to all illustration types
\captionsetup[quadro]{labelformat=customdash}
\captionsetup[figura]{labelformat=customdash}
\captionsetup[grafico]{labelformat=customdash}
\captionsetup[fotografia]{labelformat=customdash}

% Definição de funções em português de acordo com a nomeclatura da FAI
\let\secaoprimaria\section
\let\secaosecundaria\subsection
\let\secaoterciaria\subsubsection

% Configuração do BibLaTeX ==========================================================================================
\usepackage[style=authoryear, sorting=nyt, backend=bibtex]{biblatex}
\renewcommand*{\bibfont}{\raggedright}
\setlength{\bibhang}{0pt}
\setlength{\bibitemsep}{1em} % or another value like 0.5em, to control spacing between references

% Configurações para formato ABNT
\DeclareFieldFormat{title}{\textbf{#1}}
\DeclareNameAlias{sortname}{family-given}
\DeclareDelimFormat{nameyeardelim}{\addcomma\space}

% Add this new renewbibmacro to separate maintitle and subtitle with a colon
\renewbibmacro*{title+subtitle}{%
  \printfield{title}%
  \ifbibstring{\thefield{subtitle}}{}%
    {\setunit{\addcolon\space}%
     \printfield{subtitle}}}

% Formato do nome: SOBRENOME, Nome
\DeclareNameFormat{family-given}{%
  \ifgiveninits
    {\usebibmacro{name:family-given}
      {\namepartfamily}
      {\namepartgiveni}
      {\namepartprefix}
      {\namepartsuffix}}
    {\usebibmacro{name:family-given}
      {\namepartfamily}
      {\namepartgiven}
      {\namepartprefix}
      {\namepartsuffix}}%
  \usebibmacro{name:andothers}}

% Essa macro é usada pra imprimir o autor sem o ano
\newbibmacro*{author-no-year}{%
  \ifnameundef{labelname}
    {}
    {\printnames[family-given]{labelname}}%
}

% Remove the year from after the author and move it to the end
\renewbibmacro*{author/editor+others/translator+others}{%
  \ifboolexpr{
    test \ifuseauthor
    and
    not test {\ifnameundef{author}}
  }
    {%
      \printnames[family-given]{author}%
      \addspace%
      \ifnumgreater{\value{listtotal}}{\value{liststop}}
        {\mkbibemph{et\addabbrvspace al\adddot}}
        {}}%
    {\ifboolexpr{
       test \ifuseeditor
       and
       not test {\ifnameundef{editor}}
     }
       {\usebibmacro{editor+others}}
       {\usebibmacro{translator+others}}}}

\renewbibmacro*{name:family-given}[4]{%
  \ifuseprefix
    {\usebibmacro{name:delim}{#3#1}%
     \usebibmacro{name:hook}{#3#1}%
     \ifblank{#3}{}{%
       \ifcapital
         {\mkbibnameprefix{\MakeCapital{#3}}\isdot}
         {\mkbibnameprefix{#3}\isdot}%
       \ifpunctmark{'}{}{\bibnamedelimc}}%
     \mkbibnamefamily{\MakeUppercase{#1}}\isdot
     \ifblank{#2}{}{\revsdnamepunct\bibnamedelimd\mkbibnamegiven{#2}\isdot}%
     \ifblank{#4}{}{\bibnamedelimd\mkbibnamesuffix{#4}\isdot}}
    {\usebibmacro{name:delim}{#1}%
     \usebibmacro{name:hook}{#1}%
     \mkbibnamefamily{\MakeUppercase{#1}}\isdot
     \ifblank{#2}{}{\revsdnamepunct\bibnamedelimd\mkbibnamegiven{#2}\isdot}%
     \ifblank{#4}{}{\bibnamedelimd\mkbibnamesuffix{#4}\isdot}}}

% Remove parênteses do ano e coloca no final
\renewbibmacro*{date+extradate}{%
  \iffieldundef{labelyear}
    {}
    {\printtext{\printfield{labelyear}%
     \printfield{extradate}}}}

\DeclareFieldFormat{edition}{#1\adddot ed\adddot}
\DeclareFieldFormat[book]{pages}{#1\addspace p\adddot}

% Redefine o driver para livros (books)
\DeclareBibliographyDriver{book}{%
  \usebibmacro{bibindex}%
  \usebibmacro{begentry}%
  \usebibmacro{author/editor+others/translator+others} %FIXME: Isso esta imprimindo o Ano no final
  \setunit{\labelnamepunct}\newblock
  \usebibmacro{title+subtitle}%
  \newunit
  \printlist{language}%
  \newunit\newblock
  % REMOVE THIS LINE: \usebibmacro{byauthor}%
  \newunit\newblock
  \usebibmacro{byeditor+others}%
  \newunit\newblock
  \printfield[edition]{edition}%
  \newunit
  \iflistundef{location}
    {\iflistundef{publisher}
        {. [S.l.:S.n.].}
        {. [S.l.]:\setunit*{\addcolon\space}%
        \printlist{publisher}}%
    }
    {\printlist{location}%
    \iflistundef{publisher}
        {: [S.n.].}
        {\setunit*{\addcolon\space}%
        \printlist{publisher}}%
    }%
  \setunit*{\addcomma\space}%
  \usebibmacro{date+extradate}%
  \newunit\newblock
  \iftoggle{bbx:isbn}
    {\printfield{isbn}}
    {}%
  \newunit\newblock
  \usebibmacro{doi+eprint+url}%
  \newunit\newblock
  \usebibmacro{addendum+pubstate}%
  \newunit\newblock
  \printfield[pages]{pages}
  \iftoggle{bbx:related}
    {\usebibmacro{related:init}%
     \usebibmacro{related}}
    {}%
  \usebibmacro{finentry}
}

% --- Custom title format for theses (bold, no quotes) ---
\DeclareFieldFormat{thesistitle}{\mkbibbold{#1}}
\DeclareFieldFormat[thesis]{pages}{#1}

\newbibmacro*{thesis:year+pages+type+institution+location+year}{%
  \printfield{year}\adddot\space%
  \printfield{pages}\addabbrvspace{p}\adddot\space%
  \printfield{type}%
  \setunit{\addcomma\space}%
  \printlist{institution}%
  \setunit{\addcomma\space}%
  \printlist{location}%
  \setunit{\addcomma\space}%
  \printfield{year}\adddot%
}

\DeclareBibliographyDriver{thesis}{%
  \usebibmacro{bibindex}%
  \usebibmacro{begentry}%
  \usebibmacro{author/editor+others/translator+others}%
  \setunit{\labelnamepunct}\newblock
  \printfield[thesistitle]{title}%
  \iflistundef{subtitle}
    {}
    {\setunit{\addcolon\space}\printfield{subtitle}}%
  \adddot\space%
  \usebibmacro{thesis:year+pages+type+institution+location+year}%
  \usebibmacro{finentry}%
}


% Article:
% --- New macro for article journal info ---
\DeclareFieldFormat[article]{title}{#1}
\DeclareFieldFormat[article]{pages}{#1}

\newbibmacro*{article:journal+location+volume+number+month}{%
  \printfield[journaltitle]{journaltitle}%
  \iflistundef{institution}
    {}
    {\setunit{\addcomma\space}\printlist{institution}}%
  \setunit{\addcomma\space}%
  \iflistundef{location}
    {}
    {\printlist{location}\addcomma\space}%
  \printtext{v.\adddot\space}%
  \printfield{volume}%
  \setunit{\addcomma\space}%
  \printtext{n.\adddot\space}%
  \printfield{number}%
  \setunit{\addcomma\space}%
  \printtext{p.\adddot\space}%
  \printfield{pages}%
  \setunit{\addcomma\space}% FIXME: Esta mostrando um . ao invés de , depois da página
  \printfield{month}
}

% --- Redefine the driver for articles ---
\DeclareBibliographyDriver{article}{%
  \usebibmacro{bibindex}%
  \usebibmacro{begentry}%
  \usebibmacro{author/editor+others/translator+others}%
  \setunit{\labelnamepunct}\newblock
  \usebibmacro{title+subtitle}%
  \newunit\newblock
  \usebibmacro{article:journal+location+volume+number+month}%
  \newunit\newblock
  \printfield{month}\adddot\space%
  \usebibmacro{date+extradate}\adddot%
  \newunit\newblock
  \usebibmacro{doi+eprint+url}%
  \newunit\newblock
  \usebibmacro{addendum+pubstate}%
  \setunit{\bibpagerefpunct}\newblock
  \usebibmacro{pageref}%
  \usebibmacro{finentry}%
}

% --- Custom formatting for the journal title to be bold ---
\DeclareFieldFormat[article]{journaltitle}{\mkbibbold{#1}}


% Redefine o título da seção de referências para maiúsculas e centralizado
\defbibheading{bibliography}{%
  \begin{center}
    \textbf{\MakeUppercase{Referências}}
  \end{center}
}

% Separar autores na bibliografia final com ponto e vírgula.
\DeclareDelimFormat{multinamedelim}{\addsemicolon\space}
\DeclareDelimFormat{finalnamedelim}{\addsemicolon\space}
\DeclareDelimFormat{multicite}{\addcomma\space} % Isso garantirá que múltiplas citações dentro de um único comando (por exemplo, \parencite{author1, author2}) sejam separadas por uma vírgula.

\addbibresource{referencias/referencias.bib}  % Adiciona o arquivo de bibliografia (referencias/referencias.bib) como fonte para as referências bibliográficas.
% ==================================================================================================

\DeclareBibliographyCategory{exclude}%
\makeatletter
\newcommand*{\apud}[2]{%
  \addtocategory{exclude}{#1} % Referencia somente da obra consultada
  \begingroup%
    \renewcommand*{\multinamedelim}{\addcomma\addspace}%
    \renewcommand*{\finalnamedelim}{\addspace e \addspace}%
    \citeauthor{#1} 
  \endgroup%
  % if we haven't seen this key before, define a marker and insert the footnote
  \@ifundefined{apud@seen@#1}{%
    \expandafter\gdef\csname apud@seen@#1\endcsname{}%
    \footnote{\fullcite{#1}}%
  }{}%
  \space
  (%
    \citeyear{#1} \textit{apud} %
    \citeauthor{#2}%
    , \citeyear{#2}%
  )%
}
\makeatother

% Funções para Citação Direta ---------------------------------
\newenvironment{citacaodiretalonga}
{\par\addvspace{\baselineskip}% Add exactly one line space before
 \begingroup
 \list{}{\leftmargin=4cm
         \rightmargin=0pt
         \parsep=0pt
         \itemsep=0pt
         \topsep=0pt}
 \item\relax
 \singlespacing
 \fontsize{10}{12}\selectfont
 \justifying}
{\endlist
 \endgroup
 \par\addvspace{\baselineskip}}% Add exactly one line space after

% Usar:
% \begin{citacaodiretalonga}
% ...
% \end{citacaodiretalonga}
% ------------------------------------------------------------------------------------

% Alias em portugues para alguns comandos
\NewDocumentCommand{\citacaodecitacao}{ m m }{%
  \apud{#1}{#2}%
}

\NewDocumentCommand{\citacaodireta}{ o m }{%
  \IfValueT{#1}{%
    {\protected\def\mkbibnamefamily##1{\MakeTextUppercase{##1}}%
     \parencite[#1]{#2}}%
  }%
  \IfNoValueT{#1}{%
    {\protected\def\mkbibnamefamily##1{\MakeTextUppercase{##1}}%
     \parencite{#2}}%
  }%
}

% Usado quando os autores destacarem um trecho, EM NEGRITO
\NewDocumentCommand{\citacaodiretagrifonosso}{ o m }{%
  \IfValueT{#1}{%
    {\protected\def\mkbibnamefamily##1{\MakeTextUppercase{##1}}%
     \parencite[][p. #1, grifo nosso]{#2}}%
  }%
  \IfNoValueT{#1}{%
    {\protected\def\mkbibnamefamily##1{\MakeTextUppercase{##1}}%
     \parencite[][grifo nosso]{#2}}%
  }%
}

% Usado quando o autor original destacar um trecho, SUBLINHADO
\NewDocumentCommand{\citacaodiretagrifodoautor}{ o m }{%
  \IfValueT{#1}{%
    {\protected\def\mkbibnamefamily##1{\MakeTextUppercase{##1}}%
     \parencite[][p. #1, grifo do autor]{#2}}%
  }%
  \IfNoValueT{#1}{%
    {\protected\def\mkbibnamefamily##1{\MakeTextUppercase{##1}}%
     \parencite[][grifo do autor]{#2}}%
  }%
}

\NewDocumentCommand{\citacaodiretanotexto}{ o m }{%
  \IfValueT{#1}{%
    \textcite[#1]{#2}%
  }
  \IfNoValueT{#1}{%
    \textcite{#2}%
  }
}

\NewDocumentCommand{\citacaoindireta}{ m }{%
    {\protected\def\mkbibnamefamily##1{\MakeTextUppercase{##1}}%
     \parencite{#1}}%
}

\NewDocumentCommand{\citacaoindiretanotexto}{ m }{%
  \textcite{#1}%
}