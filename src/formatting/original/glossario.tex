% ======================================================================================
% Glossário - Definição do Glossário
% Comente (com % antes do texto) o conteúdo abaixo para não exibir a epigrafe
%
% Relação de palavras ou expressões técnicas de uso restrito ou de sentido obscuro, 
%    utilizadas no texto, acompanhadas das respectivas definições
%
% Glossário PODE ser usado em Monografias
% ======================================================================================

\begin{Glossario}
    % Adicione termos e definições no formato \glossario{TERMO}{DEFINIÇÃO}
    \glossario{Abreviatura}{representação de uma palavra por meio de alguma(s) de suas sílabas ou letras.}
    \glossario{Agradecimento}{são expressões de gratidão às pessoas que contribuíram efetivamente para o desenvolvimento do trabalho.}
    \glossario{Análise e Discussão dos Resultados}{parte do desenvolvimento que tem a função de analisar e interpretar os dados obtidos nos procedimentos metodológicos levando em conta o referencial teórico. Pode haver uma comparação entre a prática e a teoria, apresentando no que ambas se completam, se corroboram ou se contradizem. Os gráficos estatísticos e os relatos de testes e validações são analisados e comentados, analisando se as informações e provas obtidas confirmam ou rejeitam as hipóteses.}
    \glossario{Anexo}{texto ou documento não elaborado pelo autor, que serve de fundamentação, comprovação e complementa o trabalho. São anexos também os trabalhos do próprio autor, porém não realizados especificamente para o trabalho atual.}
    \glossario{Apêndice}{texto ou documento elaborado pelo autor para o trabalho atual, a fim de complementar sua argumentação, sem prejuízo da unidade nuclear do texto.}
    \glossario{Apuddo}{do latim, significa "citado por", "conforme", "segundo".}
    \glossario{Artigo}{e revisão: Parte de uma publicação que resume, analisa e discute informações já publicadas.}
    \glossario{Artigo}{riginal: Parte de uma publicação que apresenta temas ou abordagens originais.}
    \glossario{Autor-entidade}{quando o autor não é uma pessoa física, mas, sim, instituição(ões), organização(ões), empresa(s), comitê(s), comissão(ões), evento(s), entre outros, responsável(eis) por publicações em que não se distingue autoria pessoal.}
    \glossario{Bibliografia}{conjunto de livros sobre um assunto.}
    \glossario{Capa}{proteção externa do trabalho e sobre a qual se imprimem as informações para a sua identificação.}
\end{Glossario}