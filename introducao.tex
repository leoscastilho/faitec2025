\begin{Introducao} % ---------------------------------------------------------------
\secaoprimaria{Introdução}
% ============================================================================
% INTRODUÇÃO - EDITE O TEXTO ABAIXO
% ============================================================================
A FAI - Centro de Ensino Superior em Gestão, Tecnologia e Educação (FAI), por meio da sua biblioteca, disponibiliza este manual para normalização de trabalhos acadêmicos para os seus alunos, professores e funcionários. Este material é essencial ao uso das normas técnicas para uma apresentação correta e melhor compreensão da leitura, uma vez que um trabalho de nível superior, ou de pós-graduação, é analisado por uma banca examinadora composta por profissionais diversos, de elevado nível de conhecimentos sobre o assunto.

A normalização adotada neste manual tem como base, as normas de documentação da Associação Brasileira de Normas Técnicas (ABNT), tais como:
\begin{alinea}
    \item BNT NBR 6023:2002 - Informação e documentação - Referências - Elaboração;
    \item ABNT NBR 6024:2012 - Informação e documentação - Numeração Progressiva das seções de um documento escrito - apresentação;
    \item ABNT NBR 6027:2012 - Informação e documentação - Sumário - apresentação;
    \item ABNT NBR 6028:2003 - Informação e documentação;
    \item ABNT NBR 6034:2004 - Informação e documentação - Índice – apresentação;
    \item ABNT NBR 10520:2002 - Informação e documentação - Citação em documentos - apresentação;
    \item ABNT NBR 12225:2004 - Informação e documentação - Lombada – apresentação;
    \item ABNT NBR 14724:2011 - Informação e documentação - Trabalhos Acadêmicos - apresentação.
\end{alinea}

Devido à atualização dos padrões nacionais e internacionais, fez-se necessária mais uma revisão no manual, que por sua vez, encontra-se em sua 8ª edição. A apresentação gráfica dos textos científicos é regulamentada pela ABNT, pois segue o padrão básico internacional, o qual organiza e permite a identificação de formas e origens de textos científicos em todo o mundo (SANTOS, 2001). É importante ressaltar que em alguns casos a ABNT apresenta em suas normas algumas regras que são opcionais, permitindo que a instituição defina seus próprios critérios. Por isso, a FAI decidiu pela utilização de alguns critérios mencionados neste manual para promover a padronização e facilitar a compreensão da comunidade acadêmica acerca da realização de seus trabalhos acadêmico/científicos.
% FIM: INTRODUÇÃO ============================================================
\end{Introducao}