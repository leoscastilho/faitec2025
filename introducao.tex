\begin{Introducao} % ---------------------------------------------------------------
\secaoprimaria{Introdução}
% ============================================================================
% INTRODUÇÃO - EDITE O TEXTO ABAIXO
% ============================================================================
O esporte, enquanto ferramenta educacional e social, auxilia no desenvolvimento de crianças e adolescentes, especialmente em contextos de vulnerabilidade social. 

Diante disso, o projeto Conexão esportiva, implementado em 2016 com apoio da Escola Técnica de Eletrônica "Francisco Moreira da Costa" (ETE "FMC"), em Santa Rita do Sapucaí, Minas Gerais, tem desempenhado o papel de acolhimento e suporte aos jovens contemplando anualmente cerca de 600 alunos, matriculados no Ensino Fundamental e Médio, incentivando valores como disciplina, inclusão, respeito e trabalho em equipe.

Embora o projeto alcance resultados positivos no âmbito físico, despontando alguns jovens para times profissionais, observa-se a ausência de especialistas para demandas emocionais e comportamentais. Queixas comuns entre os professores do projeto têm sido observadas e indicam a oportunidade de atuação do profissional psicólogo em alguns tópicos relevantes que podem comprometer tanto o desenvolvimento esportivo quanto social dessas crianças, como a agressividade no esporte, que pode surgir devido a diversos fatores e se manifestar de maneira hostil ou instrumental, além da tensão competitiva e a regulação emocional, que também podem ser influenciadas pelas relações sociais que esses jovens atletas enfrentam.

Em virtude disso, este trabalho tem por objetivo aplicar técnicas de intervenção da Psicologia do Esporte a fim de trabalhar as relações interpessoais, o autocontrole, a cooperação e as habilidades sociais de jovens entre 11 e 15 anos de idade participantes do projeto Conexão Esportiva visando amenizar as queixas apresentadas pelos professores e, consequentemente, o bem estar dos jovens.
% FIM: INTRODUÇÃO ============================================================
\end{Introducao}
