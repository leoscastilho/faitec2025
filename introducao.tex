\begin{Introducao} % ---------------------------------------------------------------
\secaoprimaria{Introdução}
% ============================================================================
% INTRODUÇÃO - EDITE O TEXTO ABAIXO
% ============================================================================
O esporte, enquanto ferramenta educacional e social, auxilia no desenvolvimento de crianças e adolescentes, especialmente em contextos de vulnerabilidade social. 

Diante disso, o projeto Conexão esportiva, implementado em 2016, com apoio da Escola Técnica de Eletrônica "Francisco Moreira da Costa” - ETE "FMC”, em Santa Rita do Sapucaí, tem desempenhado o papel de acolhimento e suporte aos jovens contemplando cerca de 600 alunos, matriculados no Ensino Fundamental e Médio, incentivando valores como disciplina, inclusão, respeito e trabalho em equipe.

Por esse motivo, esta trabalho tem por objetivo Trabalhar relação, cooperação e habilidades em grupos com jovens de 11 a 15 anos participantes do Conexão esportiva; pois, embora o projeto alcance resultados positivos no âmbito físico despontando alguns jovens para times profissionais, observa-se a ausência de especialistas para demandas emocionais e comportamentais.

Em vista disso, alguns aspectos são relevantes, como a agressividade no esporte, que pode surgir devido a diversos fatores e se manifestar de maneira hostil ou instrumental. Além disso, a tensão competitiva e a regulação emocional podem ser influenciadas pelas relações sociais que esses jovens atletas enfrentam, o que pode comprometer tanto o seu desenvolvimento esportivo quanto o social.
% FIM: INTRODUÇÃO ============================================================
\end{Introducao}