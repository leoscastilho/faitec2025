% Neste apêndice, apresenta-se o formulário utilizado na coleta de dados para a presente pesquisa, o qual investigou a atuação profissional do Psicólogo do Esporte e suas percepções acerca da inclusão social no contexto esportivo.

\textbf{Nome:} Katia Rubio

\textbf{Formação Acadêmica:} Psicologia

\textbf{Número de registro do CRP:} 48080-3/06

\textbf{Local de atuação:} São Paulo

\textbf{Quais são as principais funções de um psicólogo do esporte no dia a dia?}

Cuidar de pessoas que praticam atividade física e esporte.

\textbf{Em que tipos de contextos você atua (clubes, consultórios, escolas, academias)?}

Confederação Brasileira de Voleibol, clubes e consultório.

\textbf{Qual a diferença entre psicologia clínica e psicologia do esporte na prática?}

A psicologia clínica tem seu setting definido pelo consultório. O setting da Psicologia do esporte é o local de prática do atleta, podendo ser ele privado ou público.

\textbf{De que forma o trabalho psicológico ajuda na recuperação dos indivíduos em ambiente competitivo?}

O ambiente competitivo exige objetivo na realização de uma tarefa específica, podendo ser a vitória ou a conquista de posições em rankings. Isso pode ser entendido não como recuperação, mas como busca de propósito.

\textbf{Quais são os principais desafios psicológicos enfrentados por atletas de alto rendimento?}

Determinação de objetivos; comunicação com a equipe técnica; enfrentamento dos desafios da competição; intercâmbio com o público.

\textbf{Como você trabalha questões como ansiedade pré-competição e pressão por resultados?}

Focando nos objetivos e na motivação interna; desfocando o público, o externo, as interferências no processo pessoal.

\textbf{Você percebe um aumento de atletas buscando apoio psicológico por questões emocionais?}

Sim. A superação da representação social da psicologia como a tratamento de doentes para uma prevenção em saúde mental favoreceu a busca pela psicologia.

\textbf{Como equilibrar o rendimento esportivo com o bem-estar mental?}

Transformando o esporte em mais um item da vida e não a razão de existir de alguém.

\textbf{Existe ainda tabu entre atletas e treinadores sobre buscar ajuda psicológica?}

Sim, mas já é muito menor do que foi no passado.

\textbf{Como o psicólogo do esporte lida com conflitos dentro de equipes?}

Aprimorando a comunicação entre as pessoas envolvidas com o trabalho.

\textbf{Como o esporte pode contribuir para a inclusão social de populações vulneráveis que sofrem com discriminação ou exclusão social?}

O esporte é um palco onde são encenadas muitas dramatizações sociais. Construir relações nas atividades esportivas é um treino para desenvolvê-las em outros ambientes.

\textbf{Quais os efeitos psicológicos da falta de apoio familiar ou comunitário na trajetória de jovens atletas?}

O desamparo é danoso em qualquer contexto social. No esporte, como em outras situações, a falta de apoio gera insegurança e dificuldade na realização de tarefas.

\textbf{A psicologia pode promover inclusão e pertencimento através de práticas esportivas? Como isso acontece na prática?}

A psicologia em si é um suporte para a inclusão realizada por meio da atividade esportiva e não um fim. Facilitar a comunicação; promover auto-conhecimento; desenvolver auto-estima são situações em que a psicologia colabora, juntamente com a prática esportiva, para a inclusão e pertencimento.

\textbf{Diga mais sobre a Psicologia do esporte. Algo que você acredita ser importante mas não foi mencionado no questionário.}

É preciso que as graduações em Psicologia incluam a Psicologia do Esporte como disciplina obrigatória para que mais profissionais possam atuar no ambiente esportivo com competência.

\textbf{Data e Hora da coleta:} 23/04/2025 13:30
