% Neste apêndice, apresenta-se o formulário utilizado na coleta de dados para a presente pesquisa, o qual investigou a atuação profissional do Psicólogo do Esporte e suas percepções acerca da inclusão social no contexto esportivo. A Tabela a seguir exibe as perguntas formuladas e as respectivas respostas dos profissionais que participaram do estudo.
\begin{enumerate}
    \item Nome
        \subitem Katia Rubio
    \item Formação Acadêmica
        \subitem Psicologia
    \item Número de registro do CRP
        \subitem 48080-3/06
    \item Local de atuação
        \subitem São Paulo
    \item Quais são as principais funções de um psicólogo do esporte no dia a dia?
        \subitem Cuidar de pessoas que praticam atividade física e esporte
    \item Em que tipos de contextos você atua (clubes, consultórios, escolas, academias)?
        \subitem Confederação Brasileira de Voleibol, clubes e consultório
    \item Qual a diferença entre psicologia clínica e psicologia do esporte na prática?
        \subitem A psicologia clínica tem seu setting definido pelo consultório. O setting da Psicologia do esporte é o local de prática do atleta, podendo ser ele privado ou público.
    \item De que forma o trabalho psicológico ajuda na recuperação dos indivíduos em ambiente competitivo?
        \subitem O ambiente competitivo exige objetivo na realização de uma tarefa específica, podendo ser a vitória ou a conquista de posições em rankings. Isso pode ser entendido não como recuperação, mas como busca de propósito.
    \item Quais são os principais desafios psicológicos enfrentados por atletas de alto rendimento?
        \subitem Determinação de objetivos; comunicação com a equipe técnica; enfrentamento dos desafios da competição; intercâmbio com o público
    \item Como você trabalha questões como ansiedade pré-competição e pressão por resultados?
        \subitem Focando nos objetivos e na motivação interna; desfocando o público, o externo, as interferências no processo pessoal.
    \item Você percebe um aumento de atletas buscando apoio psicológico por questões emocionais?
        \subitem Sim. A superação da representação social da psicologia como a tratamento de doentes para uma prevenção em saúde mental favoreceu a busca pela psicologia
    \item Como equilibrar o rendimento esportivo com o bem-estar mental?
        \subitem Transformando o esporte em mais um item da vida e não a razão de existir de alguém
    \item Existe ainda tabu entre atletas e treinadores sobre buscar ajuda psicológica?
        \subitem Sim, mas já é muito menor do que foi no passado
    \item Como o psicólogo do esporte lida com conflitos dentro de equipes?
        \subitem Aprimorando a comunicação e entre as pessoas envolvidas com o trabalho
    \item Como o esporte pode contribuir para a inclusão social de populações vulneráveis que sofrem com discriminação ou exclusão social?
        \subitem O esporte é uma palco onde são encenadas muitas dramatizações socias. Construir relações nas atividades esportivas é um treino para desenvolvê-las em outros ambientes.
    \item Quais os efeitos psicológicos da falta de apoio familiar ou comunitário na trajetória de jovens atletas?
        \subitem O desamparo é danoso em qualquer contexto social. No esporte, como em outras situações, a falta de apoio gera insegurança e dificuldade na realização de tarefas
    \item A psicologia pode promover inclusão e pertencimento através de práticas esportivas? Como isso acontece na prática?
        \subitem A psicologia em si é um suporte para a inclusão realizada por meio da atividade esportiva e não um fim. Facilitar a comunicação; promover auto-conhecimento; desenvolver auto-estima são situações em que a psicologia colabora, juntamente coma prática esportiva, para a inclusão e pertencimento.
    \item Diga mais sobre a Psicologia do esporte.  Algo que você acredita ser importante mas não foi mencionado no questionário.
        \subitem É preciso que as graduações em Psicologia incluam a Psicologia do Esporte como disciplina obrigatória para que mais profissionais possam atuar no ambiente esportivo com competência
    \item Data e Hora da coleta.
        \subitem 23/04/2025 13:30
\end{enumerate}