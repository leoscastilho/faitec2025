% Apresenta-se neste Apêndice o formulário utilizado na coleta de dados sobre a atuação de professores e estagiários do Projeto Conexão Esportiva de Santa Rita do Sapucaí. O instrumento contém questões abrangendo as diversas faixas etárias dos alunos atendidos pelo projeto, com o objetivo de obter um panorama sobre suas qualidades, deficiências e oportunidades de intervenção.

\textbf{Instruções de Preenchimento}

Para cada uma das afirmações apresentadas, escolha um número de 1 a 5 conforme a escala abaixo:

\begin{center}
\begin{tabular}{|c|l|}
\hline
\textbf{Escala} & \textbf{Significado} \\
\hline
1 & Discordo totalmente \\
\hline
2 & Discordo \\
\hline
3 & Nem concordo, nem discordo \\
\hline
4 & Concordo \\
\hline
5 & Concordo totalmente \\
\hline
\end{tabular}
\end{center}

\textbf{Formulário de Avaliação}

\textit{Autoconfiança e Autoestima}

\textbf{1.} Os alunos demonstram crença em seu próprio potencial e em sua capacidade de alcançar objetivos.

Avaliação: \underline{\hspace{2cm}} (1 a 5)

\textbf{2.} Os alunos se sentem confiantes para enfrentar novos desafios e sair de suas zonas de conforto.

Avaliação: \underline{\hspace{2cm}} (1 a 5)

\textit{Controle da Pressão}

\textbf{3.} Os alunos conseguem lidar bem com a pressão e com as expectativas em situações de competição.

Avaliação: \underline{\hspace{2cm}} (1 a 5)

\textbf{4.} Os alunos mantêm a calma e o foco, mesmo em momentos de alta pressão.

Avaliação: \underline{\hspace{2cm}} (1 a 5)

\textit{Motivação e Persistência}

\textbf{5.} Os alunos se mantêm motivados e persistentes mesmo quando enfrentam desafios e resultados desfavoráveis.

Avaliação: \underline{\hspace{2cm}} (1 a 5)

\textbf{6.} Os alunos mostram iniciativa para superar dificuldades e não desistem facilmente.

Avaliação: \underline{\hspace{2cm}} (1 a 5)

\textit{Concentração e Foco}

\textbf{7.} Os alunos conseguem manter a concentração durante os treinos e jogos, evitando distrações.

Avaliação: \underline{\hspace{2cm}} (1 a 5)

\textbf{8.} Os alunos demonstram foco e atenção, mesmo com fatores externos (como o uso de celular ou agitação).

Avaliação: \underline{\hspace{2cm}} (1 a 5)

\textit{Disciplina e Comprometimento}

\textbf{9.} Os alunos demonstram comprometimento e disciplina durante as atividades.

Avaliação: \underline{\hspace{2cm}} (1 a 5)

\textbf{10.} Os alunos conseguem gerenciar suas emoções, evitando que problemas pessoais ou externos interfiram no comportamento do grupo.

Avaliação: \underline{\hspace{2cm}} (1 a 5)

\textit{Controle Emocional}

\textbf{11.} Os alunos conseguem lidar com frustrações de forma saudável, sem que isso afete seu desempenho.

Avaliação: \underline{\hspace{2cm}} (1 a 5)

\textbf{12.} Os alunos demonstram estabilidade emocional, sem cobrança excessiva de si mesmos ou dos colegas.

Avaliação: \underline{\hspace{2cm}} (1 a 5)
