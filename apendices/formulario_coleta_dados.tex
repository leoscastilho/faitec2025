% Apresenta-se neste Apêndice o formulário utilizado na coleta de dados sobre a atuação de professores e estagiários do Projeto Conexão Esportiva de Santa Rita do Sapucaí. O instrumento contém questões abrangendo as diversas faixas etárias dos alunos atendidos pelo projeto, com o objetivo de obter um panorama sobre suas qualidades, deficiências e oportunidades de intervenção. A tabela a seguir detalha as perguntas formuladas e as respostas dos profissionais participantes.

Instruções: Para cada uma das afirmações, escolha um número de 1 a 5, onde:
\begin{enumerate}
    \item Discordo totalmente
    \item Discordo
    \item Nem concordo, nem discordo
    \item Concordo
    \item Concordo totalmente
\end{enumerate}

Afirmações:
\begin{enumerate}    
    \item Os alunos demonstram crença em seu próprio potencial e em sua capacidade de alcançar objetivos.
    \item Os alunos se sentem confiantes para enfrentar novos desafios e sair de suas zonas de conforto.
    \item Os alunos conseguem lidar bem com a pressão e com as expectativas em situações de competição.
    \item Os alunos mantêm a calma e o foco, mesmo em momentos de alta pressão.
    \item Os alunos se mantêm motivados e persistentes mesmo quando enfrentam desafios e resultados desfavoráveis.
    \item Os alunos mostram iniciativa para superar dificuldades e não desistem facilmente.
    \item Os alunos conseguem manter a concentração durante os treinos e jogos, evitando distrações.
    \item Os alunos demonstram foco e atenção, mesmo com fatores externos (como o uso de celular ou agitação).
    \item Os alunos demonstram comprometimento e disciplina durante as atividades.
    \item Os alunos conseguem gerenciar suas emoções, evitando que problemas pessoais ou externos interfiram no comportamento do grupo.
    \item Os alunos conseguem lidar com frustrações de forma saudável, sem que isso afete seu desempenho.
    \item Os alunos demonstram estabilidade emocional, sem cobrança excessiva de si mesmos ou dos colegas.
\end{enumerate}