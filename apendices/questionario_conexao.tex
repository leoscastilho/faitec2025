% Apresenta-se neste Apêndice o formulário utilizado na coleta de dados sobre a atuação de professores e estagiários do Projeto Conexão Esportiva de Santa Rita do Sapucaí. O instrumento contém questões abrangendo as diversas faixas etárias dos alunos atendidos pelo projeto, com o objetivo de obter um panorama sobre suas qualidades, deficiências e oportunidades de intervenção.

\textbf{1) Qual a sua formação e há quanto tempo atua no projeto Conexão Esportiva?}

\textbf{Elisa:} Estou em formação no curso de educação física. E atuo há cinco meses no Conexão.

\textbf{André Betsa:} Educação Física (Bacharelado e Licenciatura) Atua desde o início em 2016.

\textbf{Ana Júlia:} Sou Bacharel em Educação Física, com registro ativo no conselho (CREF 052992). Finalizei minha graduação no final do ano de 2023 e minha formatura foi em Fevereiro de 2024. Durante a graduação, iniciei meu estágio no Projeto Conexão Esportiva, para ser mais específica no dia 27 de setembro de 2021 onde fiquei até 21 dezembro de 2023. Precisei sair pois já havia sido ultrapassado o tempo máximo de estágio dentro do Projeto. Retornei nesse ano de 2025, em janeiro, ao Projeto Conexão Esportiva onde estou como Professora.

\textbf{2) Quais são os principais objetivos do projeto em relação ao desenvolvimento das crianças de 7 a 10 anos?}

\textbf{Elisa:} Desenvolver a coordenação motora, a psicomotricidade, apresentar os esportes e desenvolver suas habilidades.

\textbf{André Betsa:} Desenvolvimento Motor e Iniciação Esportiva, com a vivência em diversos esportes durante o ano.

\textbf{Ana Júlia:} O Projeto Conexão Esportiva é um Projeto Social Educacional, o que significa que as atividades são voltadas para o desenvolvimento global das crianças por meio do esporte. Além de ensinar fundamentos de diferentes modalidades esportivas, buscamos promover valores como respeito, trabalho em equipe, disciplina e inclusão. O foco não é apenas no desempenho físico, mas também no lado educacional e social, ajudando as crianças a crescerem como indivíduos conscientes, saudáveis e confiantes. Dentro dessa faixa etária (7 a 10) as crianças participam das turmas de Iniciação Esportiva. Lá a cada mês nós incluímos um esporte diferente e um tema transversal durante as aulas. Tudo isso é feito por meio de exercícios e brincadeiras que trabalham principalmente o desenvolvimento das capacidades motoras básicas, que são fundamentais como coordenação motora, equilíbrio, agilidade, velocidade de reação, flexibilidade, força e etc.

\textbf{3) Quais são os principais objetivos do projeto em relação ao desenvolvimento dos adolescentes e jovens acima dos 11 anos?}

\textbf{Elisa:} Desenvolver suas habilidades nos esportes específicos(vôlei, basquete, futsal e futebol de campo), através dos treinos.

\textbf{André Betsa:} Desenvolvimento do esporte de maneira específica, junto à disciplina, respeito, comprometimento...

\textbf{Ana Júlia:} Já os adolescentes e jovens, quando entram na faixa etária de 12 a 17 anos, tem a oportunidade de escolher uma modalidade específica para participar dentro do Projeto. Sendo elas Voleibol, Futebol de Campo, Futsal ou Basquete. Dentro das Modalidades Específicas o trabalho continua sendo de maneira sócio educacional, porém partimos para a formação de equipes específicas com foco competitivo onde trabalhamos o aperfeiçoamento técnico e tático: desenvolvendo habilidades específicas da modalidade esportiva, respeitando o nível e a progressão de cada aluno. O desenvolvimento físico e motor: promovendo o aprimoramento das capacidades físicas como força, resistência, velocidade, agilidade e coordenação, com treinos mais estruturados e adequados à fase de crescimento. Mas mesmo com a participação em amistosos e campeonatos sempre deixamos claro que a vivência dentro do esporte é acompanhada da formação de valores onde incentivamos a disciplina, comprometimento, responsabilidade, trabalho em equipe, respeito ao adversário e à hierarquia esportiva (regras, árbitros, treinadores).Afinal, dentro do Projeto, nosso principal objetivo em qualquer faixa etária sempre será a inclusão social e educação: usar o esporte como ferramenta para afastar os jovens de situações de risco e promover inclusão, cidadania e oportunidades de crescimento pessoal e social.

\textbf{4) Os professores e estagiários do Conexão recebem alguma orientação sobre como lidar com questões emocionais ou comportamentais dos beneficiários do projeto?}

\textbf{Elisa:} Não recebemos orientações específicas, mas temos o apoio do coordenador para conseguirmos lidar com as questões que aparecem ao longo das aulas.

\textbf{André Betsa:} Orientação não, mas sempre que tem algo que demande mais atenção, nos reunimos para solucionar da melhor maneira.

\textbf{Ana Júlia:} Nós contamos com o apoio do Coordenador do Projeto Conexão Esportiva, o Thiago Ribeiro. Ele sempre nos orienta e nos auxilia nas intervenções com questões que surgem no dia dia do Projeto.

\textbf{5) Quais desafios você percebe em relação ao comportamento, autoestima ou motivação dos jovens atletas?}

\textbf{Elisa:} O desafio maior é fazer com que eles entendam o quanto são capazes de conquistar os seus objetivos dentro do esporte. Principalmente quando estamos em algum campeonato e o placar não está favorável.

\textbf{André Betsa:} Falta de interesse (alguns fazem por obrigação), não ligar para ganhar ou perder (tanto faz), desistir muito fácil (sempre dizer que não consegue fazer) ...

\textbf{Ana Júlia:} Os principais desafios observados nos jovens atletas, na minha opinião, envolvem a falta de autoconfiança, dificuldades em lidar com frustrações e a oscilação na motivação, que muitas vezes está ligada à fase da adolescência. Também enfrentamos questões de comportamento, como indisciplina ou falta de comprometimento, que podem estar relacionadas a fatores externos, como problemas familiares ou escolares.

\textbf{6) Que benefícios você imagina que o suporte psicológico traria para o ambiente da Conexão Esportiva?}

\textbf{Elisa:} Fazer com que eles tenham motivação para continuar, mesmo com os desafios que surgem ao longo das modalidades escolhidas.

\textbf{André Betsa:} Motivação em resultados positivos, empenho em querer fazer melhor, persistir em tentar, apoiar o companheiro de forma mais adequada com o que precisam.

\textbf{Ana Júlia:} Traria benefícios significativos tanto para os alunos quanto para nós, professores. Para os jovens, ajudaria no equilíbrio emocional, na motivação e na forma de lidar com desafios e frustrações. Isso refletiria diretamente em um ambiente mais saudável, colaborativo e respeitoso para as aulas. Já para nós, os educadores, o apoio psicológico seria um reforço importante na comunicação com os alunos, no manejo de comportamentos difíceis e no entendimento das questões emocionais que impactam o rendimento e a convivência.

\textbf{7) Como as equipes lidam com a pressão durante competições importantes?}

\textbf{Elisa:} A maioria deles não conseguem lidar com essa pressão de forma sadia, atrapalhando seu rendimento dentro de quadra.

\textbf{André Betsa:} Muitos encaram o desafio e se empenham em fazer o melhor, outros ficam ansiosos, outros se deixam afetar pelo externo, alguns ficam mais estressados e cobram demais de si mesmo e dos companheiros.

\textbf{Ana Júlia:} Apesar de todo o incentivo e trabalho que realizamos, muitos jovens ainda enfrentam dificuldades para lidar com a pressão durante competições importantes. É comum que a ansiedade e o medo de errar atrapalhem o desempenho, mesmo quando estão bem preparados. Nós sabemos que isso faz parte do processo de amadurecimento esportivo e emocional, por isso seguimos trabalhando com paciência, reforçando o apoio emocional, a valorização do esforço e o aprendizado que vem com cada experiência. A presença de um suporte psicológico, por exemplo, poderia contribuir ainda mais nesse aspecto.

\textbf{8) Os atletas têm dificuldades em manter o foco durante jogos e treinos?}

\textbf{Elisa:} Se o placar não é favorável, a dificuldade é muito grande.

\textbf{André Betsa:} Varia muito de faixas etárias, sendo os mais novos menos focados e de acordo com que vão dando continuidade no projeto e subindo de categoria o foco vai aumentando, mas mesmo assim tem que ser cobrados.

\textbf{Ana Júlia:} Sim, muitos atletas ainda têm dificuldades em manter o foco durante jogos e treinos. Fatores como distrações externas, uso excessivo de celular, agitação emocional ou até mesmo a pressão por desempenho acabam interferindo na concentração.

\textbf{9) Existem estratégias para manter os atletas motivados ao longo de uma sequência de jogos?}

\textbf{Elisa:} Eu gosto de sempre conversar com eles e mostrar que mesmo estando em uma competição, estamos ali porque gostamos do que fazemos e gostamos daquele esporte. Então, temos que nos divertir.

\textbf{André Betsa:} Estratégia muda de acordo com a equipe e categoria, tentando se adequar às características de cada grupo.

\textbf{Ana Júlia:} Atualmente, não temos uma estratégia pré-estabelecida e acordada entre todos para manter a motivação dos atletas ao longo de uma sequência de jogos, mas buscamos fazer isso no dia a dia, mantendo o ambiente leve, descontraído e acolhedor. Trabalhamos muito a ideia de que somos uma equipe, quase como uma família, onde um apoia o outro, e isso tem um papel importante na motivação dos atletas. Celebramos as pequenas conquistas, reforçamos os pontos positivos e tentamos manter o clima de união e pertencimento, o que ajuda bastante a manter o ânimo mesmo nos momentos mais cansativos ou desafiadores.

\textbf{10) Existem atletas que demonstram dificuldades em lidar com o estresse, com a pressão ou com expectativas externas? Como vocês têm abordado essas situações?}

\textbf{Elisa:} Sim. Eu procuro sempre conversar e entender o lado do meu aluno para poder ver uma forma melhor de conseguir ajudar.

\textbf{André Betsa:} Sim. Tentamos mostrar a importância deles em relação ao grupo e não com qualquer cobrança que venha de fora dele.

\textbf{Ana Júlia:} Sim, alguns atletas realmente demonstram dificuldades em lidar com o estresse, a pressão e as expectativas externas. Geralmente, lidamos com essas situações por meio de conversas abertas com os próprios alunos. Criamos um espaço onde eles podem expressar suas emoções e preocupações sem julgamento, o que facilita o processo de compreensão e ajuda a encontrar soluções. Também incentivamos os atletas a focarem no esforço e no aprendizado, em vez de se concentrarem apenas nos resultados.
