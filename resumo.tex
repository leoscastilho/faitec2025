% ======================================================================================
% Resumo - Definição do texto do Resumo e das Palavras-chave
%
% Resumo DEVE ser usados em Monografias e é dispensado em Projeto de Pesquisa, Relatórios de Estágio e Trabalhos acadêmicos
% ======================================================================================

% ============================================================================
% RESUMO - EDITE O TEXTO ABAIXO ENTRE {}
% ============================================================================
\newcommand{\resumo}{
Este artigo tem como objetivo apresentar a Psicologia do Esporte e sua aplicação no projeto Conexão Esportiva, destacando a importância da atuação do psicólogo nesse contexto. O projeto atende crianças e adolescentes em fase de iniciação e formação esportiva, promovendo não apenas o desenvolvimento físico, mas também aspectos sociais. A ausência de acompanhamento psicológico representa um desafio, pois evidencia as dificuldades enfrentadas pelos jovens atletas em lidar com questões como ansiedade, motivação e tensão competitiva. A proposta visa apresentar ações práticas que envolvam alunos, professores e responsáveis, ressaltando a relevância da psicologia na formação integral dos indivíduos.
}

% ============================================================================
% PALAVRAS-CHAVE - EDITE O TEXTO ABAIXO ENTRE {} (separe com ponto e espaço)
% ============================================================================
\newcommand{\palavraschave}{
Psicologia do esporte; Intervenção; Desenvolvimento infantil; Esporte educativo.
}

% ============================================================================
% RESUMO EM INGLÊS - EDITE O TEXTO ABAIXO ENTRE {}
% ============================================================================
\newcommand{\resumoingles}{
This article aims to present Sport Psychology and its application in the project called Conexão Esportiva, highlighting the importance of the psychologist's role in this context. The project is aimed at children and adolescents in the initiation and training phases of sports, promoting not only physical development but also social aspects. The lack of psychological support poses a challenge, as it highlights the difficulties young athletes face in dealing with issues such as anxiety, motivation, and competitive tension. The proposal aims to present practical actions involving students, teachers, and guardians, emphasizing the relevance of psychology in the integral formation of individuals.
}

% ============================================================================
% PALAVRAS-CHAVE EM INGLÊS - EDITE O TEXTO ABAIXO ENTRE {} (separe com ponto e espaço)
% ============================================================================
\newcommand{\palavraschaveingles}{
Sport psychology; Intervention; Child development; Educational sport.
}

\begin{Resumo} \end{Resumo}