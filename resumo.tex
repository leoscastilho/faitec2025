% ======================================================================================
% Resumo - Definição do texto do Resumo e das Palavras-chave
%
% Resumo DEVE ser usados em Monografias e é dispensado em Projeto de Pesquisa, Relatórios de Estágio e Trabalhos acadêmicos
% ======================================================================================

% ============================================================================
% RESUMO - EDITE O TEXTO ABAIXO ENTRE {}
% ============================================================================
\newcommand{\resumo}{
Este trabalho tem como objetivo apresentar técnicas de intervenção da Psicologia do Esporte a fim de trabalhar as relações interpessoais, o autocontrole, a cooperação e as habilidades sociais de jovens entre 11 e 15 anos de idade a serem aplicadas nos participantes do projeto Conexão Esportiva.  O projeto atende crianças e adolescentes em fase de iniciação e formação esportiva, promovendo não apenas o desenvolvimento físico, mas também aspectos sociais. A ausência de acompanhamento psicológico representa um desafio, pois evidencia as dificuldades enfrentadas pelos jovens atletas em lidar com questões como ansiedade, motivação e tensão competitiva. A abordagem proposta consiste em uma intervenção por um período de quatro meses, na qual um grupo de alunos receberá sessões com técnicas de psicologia do esporte, como prática encoberta, auto-fala, relaxamento e estabelecimento de metas. A eficácia da intervenção será avaliada através da comparação de questionários aplicados aos professores antes e depois do período de intervenção. Como conclusão, o estudo reforça que a integração da psicologia no esporte vai além da melhoria de desempenho, atuando como uma poderosa ferramenta de inclusão social e desenvolvimento humano. A aplicação de técnicas psicológicas específicas pode preencher a lacuna observada e contribuir para a formação de atletas mais resilientes, emocionalmente equilibrados.
}

% ============================================================================
% PALAVRAS-CHAVE - EDITE O TEXTO ABAIXO ENTRE {} (separe com ponto e espaço)
% ============================================================================
\newcommand{\palavraschave}{
Psicologia do esporte; Intervenção; Desenvolvimento juvenil; Esporte educativo.
}

% ============================================================================
% RESUMO EM INGLÊS - EDITE O TEXTO ABAIXO ENTRE {}
% ============================================================================
\newcommand{\resumoingles}{
This study aims to present sports psychology intervention techniques to address interpersonal relationships, self-control, cooperation, and social skills in young people aged 11 to 15 which can be applied in the social project Conexão Esportiva. The project serves children and adolescents in the initial and developmental stages of sports, promoting not only physical but also social development. The lack of psychological support represents a challenge, as it highlights the difficulties young athletes face in dealing with issues such as anxiety, motivation, and competitive stress. The proposed approach consists of a four-month intervention during which a group of students will receive sessions with sports psychology techniques, such as covert practice, self-talk, relaxation, and goal setting. The effectiveness of the intervention will be evaluated by comparing questionnaires given to teachers before and after the intervention period. In conclusion, the study emphasizes that the integration of psychology in sports goes beyond performance enhancement, acting as a powerful tool for social inclusion and human development. The application of specific psychological techniques can fill the observed gap and contribute to the development of more resilient and emotionally balanced athletes.
}

% ============================================================================
% PALAVRAS-CHAVE EM INGLÊS - EDITE O TEXTO ABAIXO ENTRE {} (separe com ponto e espaço)
% ============================================================================
\newcommand{\palavraschaveingles}{
Sport psychology; Intervention; Youth development; Educational sport.
}

\begin{Resumo} \end{Resumo}
