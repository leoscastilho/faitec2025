\begin{Conclusao} % ------------------------------------------------------------------
\secaoprimaria{Conclusão}
% ============================================================================
% DESENVOLVIMENTO- EDITE O TEXTO ABAIXO. É O CONTEÚDO PRINCIPAL DO TRABALHO
% ============================================================================
A partir da análise realizada, constata-se que a atuação do psicólogo do esporte vai além da otimização do desempenho físico, contemplando aspectos fundamentais do desenvolvimento integral dos participantes.

Conforme apresentado, o projeto já desempenha um papel significativo no desenvolvimento dos jovens e até mesmo no encaminhamento de alguns para times profissionais. No entanto, a ausência de um profissional especializado para atender as demandas emocionais e comportamentais representa uma lacuna importante que deve ser preenchida pela inserção da psicologia do esporte, potencializando os resultados já alcançados.

Observou-se que o psicólogo do esporte, com sua formação abrangente e modelos diversificados de atuação, pode intervir em diferentes frentes como a regulação emocional, o manejo da agressividade e o equilíbrio do desenvolvimento. Através das técnicas de intervenção apresentadas - como a prática encoberta, a auto-fala, o relaxamento e o estabelecimento de metas - este profissional pode contribuir significativamente para o bem-estar psíquico dos jovens atletas.

A integração entre esporte e psicologia representa, portanto, uma poderosa ferramenta de inclusão social, especialmente em contextos de vulnerabilidade, promovendo não apenas habilidades físicas, mas também competências emocionais e sociais fundamentais para o desenvolvimento saudável. Ao considerar os jovens em sua totalidade, respeitando suas particularidades e limitações, o projeto Conexão Esportiva poderá ampliar seu impacto positivo, formando não apenas, atletas mais preparados, mas cidadãos mais resilientes e emocionalmente equilibrados.

Conclui-se, assim, que a implementação da psicologia do esporte no projeto conexão esportiva representa um avanço significativo na proposta de formação integral dos jovens atendidos, contribuindo para a consolidação de um modelo de iniciação esportiva que valoriza tanto o desenvolvimento físico como o bem-estar emocional dos participantes.
% FIM: CONCLUSÃO==============================================================
\end{Conclusao}