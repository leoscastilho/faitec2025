\begin{Desenvolvimento} % ------------------------------------------------------------------
% ============================================================================
% DESENVOLVIMENTO- EDITE O TEXTO ABAIXO. É O CONTEÚDO PRINCIPAL DO TRABALHO
% ============================================================================

\secaoprimaria{Quem é o psicólogo do esporte e o que ele faz?}
A Psicologia do Esporte é um campo que se junta à Antropologia, Filosofia, Sociologia do esporte, medicina, fisiologia e biomecânica do esporte em um espectro denominado Ciências do Esporte. O psicólogo que deseja atuar nesta área deve procurar desenvolver sua formação em  disciplinas que envolvam o universo do atleta. Ou seja, a formação exclusiva em psicologia, para essa área, é falha. É necessário a procura por especializações que complementam seu currículo e possibilitam uma visão universal do atleta.

Historicamente a Psicologia do Esporte se preocupava apenas com aspectos biológicos, mas agora estuda e atua em áreas como motivação, personalidade, agressão e violência, liderança, dinâmica de grupo e bem-estar de atletas, integrando enfoques sociais, educacionais e clínicos. Segundo \citacaodiretanotexto[62]{rubio1999psicologia} os modelos de atuação profissional para o psicólogo do esporte são:

\begin{alinea}
  \item clínico: profissional capacitado para atuar com equipes esportivas, clubes e/ou seleções;
  \item especialista em psicodiagnóstico: Utiliza instrumentos para avaliar o potencial e as deficiências dos atletas;
  \item conselheiro: Apoia e intervém junto aos atletas e à comissão técnica para lidar com questões coletivas ou individuais do grupo;
  \item consultor: Avalia estratégias e programas estabelecidos, buscando otimizar o rendimento;
  \item cientista - pesquisador e educador: Produz e transmite conhecimento para a área;
  \item analista: Avalia as condições do treinamento esportivo, atuando como intermediário entre atletas e comissão técnica;
  \item otimizador: Avalia o evento esportivo e organiza programas para aumentar o potencial de performance.
\end{alinea}

A diversidade de modelos reforçam a necessidade da sua formação abrangente, que por sua vez permite que o profissional atue em diversos campos do esporte, que são:
\begin{alinea}
  \item esporte de rendimento: O psicólogo atua na otimização da performance, analisando e transformando os determinantes psíquicos que interferem no rendimento de atletas e equipes;
  \item esporte escolar: O foco é a formação dos praticantes para a cidadania e o lazer, com o psicólogo compreendendo e analisando os processos de ensino, educação e socialização inerentes ao esporte;
  \item esporte recreativo: O objetivo é o bem-estar, com o psicólogo atuando na análise do comportamento recreativo de diferentes grupos em relação a motivos, interesses e atitudes;
  \item esporte de reabilitação: O trabalho é voltado para a prevenção e intervenção em pessoas com lesões decorrentes da prática esportiva ou com deficiência física e mental.
\end{alinea}

\secaoprimaria{Método}
A fim de identificar as principais queixas quanto aos jovens, foi aplicado um questionário (Apêndice \ref{apendice:questionario_conexao}) a um grupo de professores do Conexão Esportiva, das quais se destacam as seguintes:
\begin{alinea}
  \item falta de autoconfiança: os professores notam que os jovens têm dificuldades em acreditar em seu próprio potencial e em entender que são capazes de alcançar seus objetivos;
  \item dificuldade em lidar com a pressão: durante as competições, muitos atletas demonstram ansiedade e medo de errar, o que afeta seu desempenho. Há uma dificuldade geral em lidar com a pressão de resultados e expectativas externas;
  \item desmotivação e desistência: alguns jovens demonstram falta de interesse ou fazem as atividades por obrigação. Eles desistem facilmente quando enfrentam desafios e perdem a motivação, especialmente quando os resultados não são favoráveis;
  \item dificuldade de focar: fatores como distrações externas, uso de celular e agitação emocional fazem com que os atletas tenham dificuldade de manter a concentração durante treinos e jogos;
  \item problemas de comportamento: indisciplina e falta de comprometimento são desafios comportamentais que podem estar ligados a questões emocionais ou fatores externos;
  \item oscilação emocional: os jovens enfrentam dificuldades em lidar com frustrações, o que impacta seu desempenho e convivência. Eles podem ficar estressados e cobrar demais de si mesmos e dos colegas.
\end{alinea}

Identificadas as principais queixas, foi formulado um questionário (ver Apêndice \ref{apendice:formulario_coleta_dados}) com 12 perguntas que utilizam a escala Likert para avaliar a percepção dos professores quanto ao comportamento dos alunos.


\secaoprimaria{Fundamentação teórica}

\secaosecundaria{Benefícios da Psicologia do Esporte}
A Psicologia do Esporte é crucial para analisar emoções e comportamentos no esporte, pois a prática esportiva associada ao acompanhamento psicológico desenvolvem autoestima, resiliência e bem-estar, diminui o Burnout e melhora o manejo emocional. Traz beneficios à saúde física e mental, a qualidade de vida e a inclusão social, melhorando a performance de atletas, reduzindo estresse, ansiedade e sintomas de doenças psíquicas,

O Brasil, apesar do crescimento global, enfrenta desafios como: resistência institucional e falta de investimento especialmente na educação de base.

\secaosecundaria{O esporte como inclusão social}
Em entrevista, Rubio afirma que "O esporte é um palco onde são encenadas muitas dramatizações sociais. Construir relações nas atividades esportivas é um treino para desenvolvê-las em outros ambientes.” (APÊNDICE \ref{apendice:questionario_rubio})

Em contextos de vulnerabilidade social, é comum que crianças e adolescentes estejam expostos a diversos fatores hostis, como violência, negligência e a inacessibilidade a recursos básicos, e essas situações, somadas ao poder de ambientes desestruturados, podem impactar negativamente o desenvolvimento físico, emocional e cognitivo dos jovens. É nesse cenário que muitas crianças e adolescentes chegam aos projetos de iniciação esportiva, frequentemente apresentando sintomas negativos influenciados pelo meio em que vivem.

Ao analisar o esporte de forma isolada, é evidente sua contribuição para o aspecto físico do indivíduo, pois o corpo humano, de forma inerente, produz e secreta hormônios que promovem o bem-estar e a homeostase, como destacado pelo Conselho Regional de Educação Física ao abordar os impactos fisiológicos da atividade física \citacaoindireta{joaoefigueirajunior2019}.

\citacaodiretanotexto[5]{sancheserubio2011} afirmam que "se essa prática for conduzida de acordo com as premissas da educação pelo esporte, ela pode contribuir imensamente para o desenvolvimento saudável do praticante.”.

Entretanto, ao considerar fatores como a regulação emocional, habilidades sociais e o histórico conturbado desses jovens, é possível que o desempenho e a evolução no esporte escolhido apresentem prejuízos. Nessas circunstâncias, a presença de um profissional habilitado para cuidar da saúde mental torna-se imprescindível, não apenas para melhorar o rendimento na atividade esportiva, mas para cuidar do ser humano e possibilitar a inclusão dos jovens em uma nova realidade.

Nesse contexto, esporte e psicologia esportiva se interligam em projetos como o Programa Atleta Cidadão, que contempla todo o território nacional e Secretarias de Esportes que desenvolvem projetos que atingem núcleos menores de indivíduos, ou seja, torna-se importante  ressaltar que o psicólogo do esporte não atua sozinho, trabalhando em conjunto com organizações, comissões técnicas ou com qualquer atleta, profissional ou não, que deseje melhorar seu desempenho. A psicologia esportiva pode promover inclusão e pertencimento por meio de práticas personalizadas, adaptadas a cada realidade, física ou social, favorecendo o contato, acolhimento e o apoio aos envolvidos.

Apesar desses avanços, a cidade de Santa Rita do Sapucaí não conta com psicólogos especializados na área esportiva, e por isso projetos sociais de iniciação esportiva, como o Projeto Conexão Esportiva, ainda carecem desse suporte. Segundo o professor André, o apoio psicológico pode trazer uma grande contribuição ao projeto, como "motivação em resultados positivos, empenho em querer fazer melhor, persistir em tentar apoiar o companheiro de forma mais adequada com o que precisam” (BETSA, 2025, Apêndice B).

\secaosecundaria{Tensão competitiva e regulação emocional}
No contexto esportivo, especialmente entre crianças e adolescentes, o rendimento dos atletas não depende apenas de suas habilidades físicas e técnicas, mas também de aspectos emocionais que influenciam diretamente sua atuação. Emoções como ansiedade, raiva ou entusiasmo podem tanto impulsionar quanto comprometer o desempenho.

Desse modo, a teoria dos quatro aspectos de Nitsch fornece uma estrutura eficaz para compreender a relação entre emoção e rendimento, demonstrando como a vivência emocional interfere nas interações sociais, nos treinos e nas competições \citacaoindireta{samulski2002psicologia}.

A tensão competitiva, nesse sentido, pode ser entendida como resultado de emoções que exercem duas funções principais: a função de organização, orientação e controle das ações, e a função energética, relacionada à ativação do organismo. Por exemplo, mesmo um atleta desmotivado pode demonstrar mais envolvimento nos treinos do que outro ainda mais desmotivado, o que revela a complexidade da influência emocional sobre o comportamento esportivo.

A partir dessa observação, reforça-se a ideia de que as emoções mantêm uma conexão estreita com o rendimento. \citacaoindiretanotexto{samulski2002psicologia} afirma que para compreender essa relação é necessário considerar os aspectos apontados por Nitsch:
\begin{alinea}
  \item a emoção: qualidade, intensidade e forma de manifestação;
  \item a situação: relação pessoa-meio e ambiente-tarefa;
  \item o efeito causado: positivo ou negativo;
  \item o rendimento: qualidade, quantidade, gasto e efeito.  
\end{alinea}

É importante perceber que essa ligação entre rendimento e emoção não se limita apenas à atuação em campo, mas também se estende aos treinos e às situações de socialização às quais pré-adolescentes e adolescentes estão expostos.

Um exemplo disso pode ser observado quando um adolescente tem uma pequena discussão com um colega de equipe fora de campo, e durante o treinamento ou até mesmo em uma competição. Nesses casos, a comunicação entre os dois tende a falhar devido à interferência emocional, seja raiva, tristeza ou culpa. Como resultado, o rendimento de ambos pode ser comprometido, não apenas pela falta de apoio mútuo, mas também pela dificuldade em manter a neutralidade emocional durante a partida.

Nesse cenário, os quatro aspectos são facilmente identificáveis: a emoção é negativa; a intensidade pode ser elevada; a manifestação ocorre por meio da perda de concentração. A situação envolve o relacionamento interpessoal (pessoa-meio) e a exigência da tarefa (ambiente-tarefa). O efeito causado é negativo, gerando frustração e fazendo com que o foco no jogo diminua. Assim, o rendimento torna-se limitado em quantidade e em qualidade (SAMULSKI, 2002).

É evidente, portanto, que emoções positivas e negativas fazem parte do cotidiano esportivo de pré-adolescentes e adolescentes, influenciadas por três estados emocionais descritos por Samulski (2002): o estado de febre (alta excitação e tremores corporais), o estado de apatia (inibição central, baixa excitabilidade e fadiga) e o estado ótimo de ativação (equilíbrio entre excitação e inibição, com intensidade fisiológica ideal). Esses estados podem ser vivenciados em sequência: inicialmente, há uma grande animação com o início do jogo; em seguida, surge a apatia diante das dificuldades e do cansaço; e, com o tempo e a experiência, o atleta tende a alcançar um equilíbrio emocional, essencial para o bom desempenho esportivo.

Diante disso, a atuação de um psicólogo do esporte torna-se fundamental. Esse profissional pode auxiliar na resolução de conflitos internos e externos vivenciados por jovens atletas, fortalecendo aspectos positivos de forma adequada, sem gerar perdas de motivação ou excesso de excitação. O psicólogo atua para promover equilíbrio emocional, evitando pressões, medos e lidando com questões como estresse, ansiedade, excitação e apatia. Quando as emoções estão equilibradas, o rendimento tende a melhorar significativamente e a tensão se torna menos presente, contribuindo para um ambiente esportivo mais saudável e produtivo.

\secaosecundaria{A agressividade no esporte}
Outro aspecto a ser observado e possivelmente trabalhado por um profissional da psicologia dentro da Conexão Esportiva é a agressividade. Compreender os aspectos da agressividade no esporte é fundamental para promover um ambiente saudável de desenvolvimento. A agressividade pode se manifestar de forma hostil, com a intenção de causar dano, ou instrumental, como meio para alcançar um objetivo competitivo. A maioria dos jovens atletas tende a rejeitar comportamentos hostis, reconhecendo a importância de uma “agressividade positiva” ligada ao esforço e à superação. \citacaoindireta{pinheiro2024}

Fatores externos, como a frustração de familiares e a pressão da torcida, quando presente, muitas vezes influenciam negativamente o comportamento das crianças, elevando os níveis de estresse e potencializando reações agressivas. A família e os treinadores são referências fundamentais na formação dos atletas, podendo tanto reforçar atitudes negativas quanto servir de exemplo de autocontrole e respeito. 

\begin{citacaodiretalonga}
"Pode-se considerar que quando uma mãe ou pai assiste a uma partida competitiva do seu filho, esse indivíduo está confiante em relação à vitória e alimentando expectativas otimistas dentro de si, e a mera esperança de um final favorável já é razão para um aumento das tendências hostis" \citacaodecitacao{cabral2020}{pinheiro2024}.
\end{citacaodiretalonga}

Nesse cenário, a presença de um psicólogo dentro do programa Conexão Esportiva pode trazer contribuições significativas. O profissional pode desenvolver ações com pais, treinadores e adolescentes, promovendo valores como empatia, controle emocional e resolução de conflitos. Atividades em grupo, oficinas e atendimentos individuais podem ajudar os jovens a lidarem melhor com frustrações, expectativas e rivalidades naturais do esporte.

Além disso, o psicólogo pode mediar conflitos familiares nas arquibancadas e conscientizar sobre os impactos emocionais das atitudes dos adultos nas adolescentes. Com foco no bem-estar e no desenvolvimento integral dos participantes, a atuação psicológica no esporte se mostra essencial para fortalecer um ambiente educativo, acolhedor e transformador.

\secaoprimaria{Proposta de intervenção}
Trabalhar com Psicologia do Esporte envolve desafios que mudam de acordo com o “cliente”.  Pois envolve o trabalho com atletas que trabalham sozinhos, em duplas e até mesmo em grandes times. Cada tipo de atuação vai demandar uma visão diferente de trabalho. 

\citacaoindiretanotexto{scala2000proposta} propõe um trabalho desenvolvido através de programas de treinamento, que buscam a melhora do desempenho esportivo se baseia nas técnicas mais encontradas na literatura que quando combinadas têm melhor efeito no treinamento dos atletas. 

Diferente da terapia, que se orienta pelas queixas do paciente, na Psicologia do Esporte o objetivo do trabalho é melhorar o rendimento esportivo, portanto esses programas de treinamento visam instalar comportamentos específicos que vão resultar na melhora  de desempenho do atleta. Na literatura sobre psicologia do esporte as técnicas mais utilizadas são:

\begin{alinea}
  \item Prática Encoberta ou Treino mental - técnica na qual o atleta treina através da sua imaginação. "Imaginar, para Skinner(1974) é: `ver algo na ausência da coisa vista, é presumivelmente uma questão de fazer aquilo que se faria quando o que se vê está presente.`" \citacaodecitacao{skinner1993}{scala2000proposta}. Estudos mostram que a prática encoberta promove alterações cerebrais, musculares e do sistema nervoso autônomo como se o atleta estivesse fazendo o movimento em si;
  \item auto-fala - técnica na qual o sujeito repete uma frase ou uma palavra durante o desempenho de uma tarefa, sua função é controlar as emoções levando o atleta a ter mais controle na solução de problemas, na concentração e também evitar se sensibilizar com pensamentos negativos;
  \item relaxamento - as técnicas de relaxamento permitem o alívio de tensões corporais, controle da ativação do sistema nervoso central, para que o atleta não tenha excesso ou falta de energia para atuar;
  \item estabelecimento de metas -  estratégia que permite melhorar o rendimento do atleta através do planejamento das metas que ele quer alcançar. Essas metas devem ser definidas de acordo com a situação atual do atleta e devem ser estabelecidas com critérios concretos, para diminuir a possibilidade de fracasso, sempre colocando prazos para cumprí-las.
\end{alinea}

Estas quatro técnicas combinadas promovem um aumento do rendimento dos atletas e também um maior comprometimento consigo e com o grupo em que atua. Não é necessário utilizar todas juntas, pode-se combinar duas ou três dependendo da situação, por exemplo:
\begin{alinea}
  \item prática encoberta + relaxamento: facilita a aprendizagem, pois facilita imaginar se o atleta encontra-se relaxado;
  \item prática encoberta + autofala: durante o treino mental se implementa a auto fala treinando o cérebro para respostas mais rápidas quando utilizadas nos treinos e/ou competições.
\end{alinea}

\secaoprimaria{Considerações finais}




% FIM: DESENVOLVIMENTO ================================================================
\end{Desenvolvimento}


